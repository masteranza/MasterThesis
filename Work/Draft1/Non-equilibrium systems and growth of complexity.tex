% !TEX encoding = Mac Central European Roman
\documentclass[a4paper,12pt]{article}

\usepackage{amsmath}
\usepackage{amssymb}
\usepackage{mathtools}
\usepackage{verbatim}
\usepackage[polish]{babel}
\usepackage[T1]{fontenc}
\usepackage[macce]{inputenc}
%\usepackage[latin2]{inputenc}

\usepackage{xcolor}
\definecolor{lightgray}{gray}{0.90}
\newtheorem{theorem}{Theorem}
\usepackage{framed}
\renewenvironment{leftbar}[1][\hsize]
{% 
\def\FrameCommand 
{%

    {\hspace{-3pt}\color{black}\vrule width 3pt}%
    \hspace{0pt}%must no space.
    \fboxsep=\FrameSep\colorbox{lightgray}%
}%
\MakeFramed{\hsize#1\advance\hsize-\width\FrameRestore}%
}
{\endMakeFramed}
\setlength{\FrameSep}{0pt}

\textwidth=16.5 truecm
\textheight=25 truecm
\hoffset=-2.5 truecm
\voffset=-2 truecm

\def\baselinestretch{1.2}

\pagestyle{empty}

\begin{document}

\title{Non-equilibrium systems and growth of complexity}

\author{Micha{\l } Mandrysz \\
Instytut Fizyki, Uniwersytet Jagiello�ski, ul. {\L }ojasiewicza
11, 30-348 Krak�w, Polska }

\maketitle

\section{General properties of irreversible processes}

The term "irreversible process" is synonymous with a specific direction of time.
To our best knowledge the direction of time is an emergent phenomena. 
Isolated thermodynamic systems evolve in the direction of more probable states i.e. growing entropy.
When they finally reach that state the direction of time ceases to exist for the whole system.
Therefore in the description of irreversible processes the history of the systems evolution becomes relevant.

\end{document}