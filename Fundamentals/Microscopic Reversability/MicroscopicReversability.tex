%% AMS-LaTeX Created with the Wolfram Language for Students - Personal Use Only : www.wolfram.com

\documentclass{article}
\usepackage{amsmath, amssymb, graphics, setspace}

\newcommand{\mathsym}[1]{{}}
\newcommand{\unicode}[1]{{}}

\begin{document}

\title{Microscopic Reversability}
\author{}
\date{}
\maketitle

\section*{Probability distributions}

Microscopic reversibility is a relation between probability ratios of trajectories and the heat.

\(E\) designates the vector of energy states, \(E_x\) the energy for state \(x\in \{1,2,\ldots ,N\}\) \\
If these state energies do not vary with time then the stationary probability distribution \(\pi _x\) is given by the canonical ensemble of equilibrium
statistical mechanics

\(\pi _x=\rho (x|\beta ,E)=\frac{e^{-\beta \, E_x}}{\sum _x e^{-\beta \, E_x}}=\exp \left(\beta \, F-\beta \, E_x\right)\)

\(F(\beta ,E)=-\beta ^{-1}\ln  \sum _x  \exp \left(-\beta \, E_x\right)=-\beta ^{-1}\ln  Z\)

The distribution \(\rho (x|\beta ,E)\), is the probability of state \(x\) given the temperature of the heat bath and the energies of all the states.

We adopt the convention that when there is insufficient information to uniquely determine the probability distribution, then the maximum entropy
distribution should be assumed

In contrast to an equilibrium ensemble, the probability distribution of a non-equilibrium ensemble is not determined solely by the external constraints,
but explicitly depends on the dynamics and history of the system.

\section*{Markov chains}

\subsection*{Transfer matrix}

\(M(t)_{x(t+1)x(t)}\equiv P(x(t)\to x(t+1))\)

all elements must be non$--$negative

all columns must sum to one (normalization of probabilities)

Single time step dynamics can be written as

\(\rho (t+1)=M(t)\rho (t)\)

A vector of transition matrices \(M=(M(0),M(1),\ldots ,M(\tau -1))\) completely determines the dynamics of the system.

The state energies are always finite

\subsubsection*{Balanced transition}

We call a transition matrix balanced, or say that the equilibrium distribution \(\pi (t)\) is an invariant distribution of \(M(t)\).

\(\pi (t)=M(t)\pi (t)\)

If the system is already in equilibrium and the system is unperturbed, then it must remain in equilibrium

\subsubsection*{Detailed balance transition}

\(M(t)_{\text{ij}}\pi (t)_j=M(t)_{\text{ji}}\pi (t)_i\)

Many Monte Carlo simulations are detailed balanced, but it{'}s NOT a necessity in such simulations and for microscopic reversibility

\section*{Microscopic picture of work and heat}

Work $--$ change in states energy levels

Heat $--$ changes between states (here the convention is that it{'}s the heat \(Q\) that flows into the system from the attached heat bath)

\(Q[x]=\sum _{t=0}^{\tau -1} \left[E(t)_{x(t+1)}-E(t)_{x(t)}\right]\)

\(W[x]=\sum _{t=0}^{\tau -1} \left[E(t+1)_{x(t+1)}-E(t)_{x(t+1)}\right]\)

\(\text{$\Delta $E}=E(\tau )_{x(\tau )}-E(0)_{x(0)}=W+Q\)

\subsection*{Reversible work}

\(W_r=\text{$\Delta $F}=F(\beta ,E(\tau ))-F(\beta ,E(0))\)

It{'}s a state function!

\subsection*{Dissipative work}

\(W_d[x]=W[x]-W_r\)

\section*{Time reversal}

Can be implemented by reversal of dynamics or permutations of the state space (inverting momenta of all particles)

\subsection*{Reversal of dynamics}

\(\hat{x}(t)=x(\tau -t)\)

\(\hat{E}(t)=E(\tau -t)\)

Consider a time homogenous Markov chain (chain with a time$--$independent transition matrix). Let \(\pi\) be the invariant distribution of the transition
matrix \(\fbox{$(\ref{invDistM}.\ref{invDistM})$}\).\\
If the system is in equilibrium then a time reversal should have no effect on that distribution.\\
Probability \(i\to j\) in forward chain should be equal to probability of \(j\to i\) in the time reversed chain. So it follows that:

\(\hat{M}_{\text{ji}}\pi _i=M_{\text{ij}}\pi _j, \text{for} \text{all} i,j\)

\(\hat{M}\) is called the reversal of \(M\) or the \(\pi\)$--$dual of \(M\).

This can be solved for \(\hat{M}\) using a simple trick

\(\hat{M}=\text{diag}(\pi )M\mathsf{T}\text{diag}(\pi )^{-1}\)

Which works because:

\(\hat{M}_{\text{ji}}\pi _i=\delta _{\text{jk}}\pi _k(M\mathsf{T})_{\text{kl}}\left(\frac{1}{\pi }\right)_l\delta _{\text{li}}\pi _i=\pi _j(M\mathsf{T})_{\text{ji}}\frac{1}{\pi
_i}\pi _i=M_{\text{ij}}\pi _j\)

If the transition matrix obeys the detailed balance \(\fbox{$(\ref{detailedBalance}.\ref{detailedBalance})$}\) then \(\hat{M}=M\).

For non$--$homogeneous chain we have

\(\hat{M}=\text{diag}(\pi (\tau -t))M(\tau -t)\mathsf{T}\text{diag}(\pi (\tau -t))^{-1}\)

For reverse transitions

\(\hat{\rho }(t)=\hat{M}(t)\hat{\rho }(t-1)\)

\(Q[x]=-Q\left[\hat{x}\right]\)

\(W[x]=-W\left[\hat{x}\right]\)

\(W_d[x]=-W_d\left[\hat{x}\right]\)

\section*{Forward/reverse trajectories relation}

Let \(\mathcal{P}[x|x(0),M]\) be the probability of observing the trajectory \(x\), given that the system started in state \(x(0)\).\\
The probability of the corresponding reversed path is \(\hat{\mathcal{P}}\left[\hat{x}|\hat{x}(0),\hat{M}\right]\). The ratio of those path probabilities
is a simple function of the heat exchanged with the bath

\(\frac{\mathcal{P}[x|x(0),M]}{\hat{\mathcal{P}}\left[\hat{x}|\hat{x}(0),\hat{M}\right]}=e^{-\beta \, Q[x]}\)

$\quad $First we expand the path probability as a product of single time step probabilities (Markovian dynamics)\\
 \(\frac{\mathcal{P}[x|x(0),M]}{\hat{\mathcal{P}}\left[\hat{x}|\hat{x}(0),\hat{M}\right]}=\frac{\prod _{t=0}^{\tau -1} P(x(t)\to x(t+1))}{\prod _{t'=0}^{\tau
-1} P\left(\hat{x}\left(t'\right)\to \hat{x}\left(t'+1\right)\right)}=\frac{M(0)_{x(1)x(0)}M(1)_{x(2)x(1)}\ldots  M(\tau -1)_{x(\tau )x(\tau -1)}}{\hat{M}(1)_{\hat{x}(1)\hat{x}(0)}\hat{M}(2)_{\hat{x}(2)\hat{x}(1)}\ldots
 \hat{M}(\tau )_{\hat{x}(\tau )\hat{x}(\tau -1)}}\overset{\hat{x}\to x}{=}\frac{M(0)_{x(1)x(0)}M(1)_{x(2)x(1)}\ldots  M(\tau -1)_{x(\tau )x(\tau
-1)}}{\hat{M}(1)_{x(\tau -1)x(\tau )}\hat{M}(2)_{x(\tau -2)x(\tau -1)}\ldots  \hat{M}(\tau )_{x(0)x(1)}}\overset{\fbox{$(\ref{nonhomoInvChain}.\ref{nonhomoInvChain})$}}{=}\left(M(0)_{x(1)x(0)}M(1)_{x(2)x(1)}\ldots
 M(\tau -1)_{x(\tau )x(\tau -1)}\right)/\left(\pi (\tau -1)M(\tau -1)_{x(\tau )x(\tau -1)}\pi (\tau -1)^{-1}\pi (\tau -2)M(\tau -2)_{x(\tau -1)x(\tau
-2)}\pi (\tau -2)^{-1}\ldots \text{ $\, $}\pi (0)M(0)_{x(1)x(0)}\pi (0)^{-1}\right)\overset{M's \text{are} \text{numbers}}{=}\prod _{t=0}^{\tau -1}
\frac{\pi (t)_{x(t+1)}}{\pi (t)_{x(t)}}\overset{\fbox{$(\ref{probDist}.\ref{probDist})$}}{=}\prod _{t=0}^{\tau -1} \frac{\rho (t+1,\beta ,E(t))}{\rho
(t+1|\beta ,E(t))}=\exp \left\{-\beta \sum _{t=0}^{\tau -1} \left[E(t)_{x(t+1)}-E(t)_{x(t)}\right]\right\}\overset{\fbox{$(\ref{defHeat}.\ref{defHeat})$}}{=}\exp
\{-\beta \, Q[x]\}\)\\
 

Essential assumptions made along the way:\\
$--$ accessible energies are always finite\\
$--$ dynamics are Markovian\\
$--$ unperturbed preserve the equilibrium distribution

Crooks, claims {``}those conditions are valid independently of the strength of the perturbation, or the distance of the ensemble from equilibrium{''}
are they?\\
Looks like they could since \(E(t)\) depends on \(t\)

More commonly we see a relation in which the dissipated heat is used instead of heat that entered the system

\(\frac{\hat{\mathcal{P}}\left[\hat{x}|\hat{x}(0),\hat{M}\right]}{\mathcal{P}[x|x(0),M]}=e^{-\beta \, Q[x]}\)

\section*{Generalization to continuous$--$time Markov chains}

Assumptions:

there are only a finite number of state transitions in finite time

process is {``}right$--$continuous{''} 

\(x(s)=x(t) \text{for} t\leq s\leq t+\epsilon\)

\subsection*{Transition rate matrix}

\(Q_{\text{ij}}\geq 0, i\neq j\)

\(\sum _i Q_{\text{ij}}=0, \forall i\)

The negatives of the diagonal elements \(-Q_{\text{ii}}\) are the rate of leaving state \(i\), and the off$--$diagonal elements \(Q_{\text{ij}}\)
give the rate of going from state \(j\) to state \(i\).

For homogeneous Markov process the finite time transition matrix, \(M\), can be obtained from a matrix exponential of the corresponding \(Q\) matrix

\(M=e^{\tau \, Q}=\sum _{k=0}^{\infty } \frac{(\tau \, Q)^k}{k!}\)

Because \(M\) is a polynomial in \(Q\) it follows that \(Q\) and \(M\) always have the same invariant distributions.\\
The elements \(M_{\text{ij}}\) are the probabilities of the system being in state \(i\) at time \(\tau\), given that it was in state \(j\) at time
\(0\).

This representation therefore loses information about the path that was taken between state \(j\) and state \(i\)

\subsection*{Trajectories approach}

Alternative approach is to consider explicitly the probability of a trajectory. The total probability can be broken into a product of a single step
transition probabilities. These in turn can be split into a holding time probability and a jump probability.

\(\mathcal{P}[x|x(0),Q]=\prod _{j=0}^{J-1} P_H\left(x\left(t_j\right)\right)P_J\left(x\left(t_j\right)\to x\left(t_{j+1}\right)\right)\)

The holding time probability, \(P_H\left(x\left(t_j\right)\right)\), is the probability of holding in state \(x\left(t_j\right)\) for a time \(t_{j+1}-t_j\),
before making a jump to the next state.\\
For a non$--$homogeneous chain we can write the holding time probabilities in terms of the diagonal elements of \(Q(t)\),

\(P_H\left(x\left(t_j\right)\right)=\exp \left\{\underset{t_j}{\overset{t_{j+1}}{\int }}Q\left(t'\right)_{x\left(t_j\right)x\left(t_j\right)}dt'\right\}\)

The jump probabilities, are probabilities of making the specified transition given that some jump has occurred

\(P_J\left(x\left(t_j\right)\to x\left(t_{j+1}\right)\right)=-Q\left(t_{j+1}\right)_{x\left(t_j+1\right)x\left(t_j\right)}/Q_{x\left(t_j\right)x\left(t_j\right)}\)

Note that the jump from state \(x\left(t_j\right)\) to \(x\left(t_{j+1}\right)\) occurs at time \(t_{j+1}\), and that the jump probabilities depend
only on the transition rate matrix at that time.

\subsection*{Time reversal}

\(\hat{t}_j=\tau -t_{J-j}, \forall j\)

Equilibrium ensemble should again be unaffected by a time reversal, and that the probability of a transition in equilibrium should be the same as
the probability of reverse transition in the reverse dynamics. The time reversal of the transition rate matrix is then identical to the time reversal
of the discrete time transition matrix.

\(\hat{Q}(t)=\text{diag}(\pi (\tau -t))Q(\tau -t)\mathsf{T}\text{diag}(\pi (\tau -t))^{-1}\)

Recall that \(\pi (t)\) is the invariant distribution of \(Q(t)\), which is identical to the probability distribution defined by equilibrium statistical
mechanics, \(\pi (t)=\rho (x|\beta ,E(t))\).

This transformation does not alter the diagonal elements of \(Q\), so the holding time probabilities of the reverse path are the same as those in
the forward path.

The jump probabilities of the forward path can be simply related to the jump probabilities of the reverse path via the time reversal operation. With
these observations and definitions it is now possible to show that continuous time Markovian dynamics is also microscopically reversible.

\(\hat{\mathcal{P}}\left[\hat{x}|\hat{x}(0),\hat{Q}\right]=\prod _{j=0}^{J-1} P_H\left(\hat{x}\left(\hat{t}_{j+1}\right)\right)P_J\left(\hat{x}\left(\hat{t}_j\right)\to
\hat{x}\left(\hat{t}_{j+1}\right)\right)=\\
\prod _{j=0}^{J-1} P_H\left(x\left(t_j\right)\right)P_J\left(x\left(t_j\right)\to x\left(t_{j+1}\right)\right)\frac{\rho \left(\left.x\left(t_j\right)\right|\beta
,E\left(t_j\right)\right)}{\rho \left(\left.x\left(t_{j+1}\right)\right|\beta ,E\left(t_j\right)\right)}=\\
\mathcal{P}[x|x(0),Q]\exp \{+\beta \, Q[x]\}\)

\section*{Generalization to continuous$--$time and space Markov processes}

Note that some measure theoretic subtleties are being ignored. For example, the state of the system should strictly speaking, be represented by an
element of a Borel set, rather than an arbitrary element of \(\mathbb{R}^n\).

Let \(x(t)\) be the state of the system at time \(t\). The phase space probability density is \(\rho (x,t)\). The time evolution of the probability
density can be described by an operator \(U\)

\(\rho \left(x,t_2\right)=U\left(t_2,t_1\right)\rho \left(x,t_1\right)=\int P\left(x,t_2|x',t_1\right)\rho \left(x',t_1\right)dx', t_2\geq t_1\)

We assume that statistical mechanics correctly describes the equilibrium behavior of the system. It follows that for an unperturbed system the propagator
must preserve the appropriate equilibrium distribution.

\(\rho (x|\beta ,E)=\int P\left(x,t\left|x'\right.,t'\right)\rho \left(\left.x'\right|\beta ,E\right)dx'\)

The probability density of the path \(x(t)\), given the set of evolution operators \(U\) is

\(\mathit{p}[x(t)|U]=\prod _{j=0}^{J-1} P\left(x,\left.t_{j+1}\right|x,t_j\right)\)

In this derivation, it was assumed that the state energies only changed in discrete steps.

\section*{Example}

\begin{doublespace}
\noindent\(\pmb{M=\{\{0,0,2./3,1./3,0,0\},\{2./3,0,0,0,1./3,0\},\{0,2./3,0,0,0,1./3\},}\\
\pmb{\{1./3,0,0,0,2./3,0\},\{0,1./3,0,0,0,2./3\},\{0,0,1./3,2./3,0,0\}\};}\)
\end{doublespace}

\begin{doublespace}
\noindent\(\pmb{x=\text{Array}[X,6];}\\
\pmb{\text{Normalize}[\text{First}@(x\text{/.}\text{Solve}[x\text{==}M.x,x]\text{/.}\{X[1]\to 1\}),\text{Total}]}\)
\end{doublespace}

\begin{doublespace}
\noindent\(\{0.166667,0.166667,0.166667,0.166667,0.166667,0.166667\}\)
\end{doublespace}

\begin{doublespace}
\noindent\(\pmb{Q=\text{MatrixLog}[M]}\)
\end{doublespace}

\begin{doublespace}
\noindent\(\{\{-0.549306,-1.54593,1.54593,1.21372,-0.332208,-0.332208\},\{1.54593,-0.549306,-1.54593,-0.332208,1.21372,-0.332208\},\{-1.54593,1.54593,-0.549306,-0.332208,-0.332208,1.21372\},\{1.21372,-0.332208,-0.332208,-0.549306,1.54593,-1.54593\},\{-0.332208,1.21372,-0.332208,-1.54593,-0.549306,1.54593\},\{-0.332208,-0.332208,1.21372,1.54593,-1.54593,-0.549306\}\}\)
\end{doublespace}

\begin{doublespace}
\noindent\(\pmb{\text{Manipulate}\left[Q+\text{Sum}\left[\frac{\text{MatrixPower}[Q,k]}{k!},\{k,1,N\}\right]\text{//}\text{MatrixForm},\{\{N,1\},1,100\}\right]}\)
\end{doublespace}

\begin{doublespace}
\noindent\(\)
\end{doublespace}

\end{document}
