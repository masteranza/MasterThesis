%% AMS-LaTeX Created with the Wolfram Language for Students - Personal Use Only : www.wolfram.com

\documentclass{article}
\usepackage{amsmath, amssymb, graphics, setspace}

\newcommand{\mathsym}[1]{{}}
\newcommand{\unicode}[1]{{}}

\begin{document}

\title{Schr{\" o}dinger $\&$ Prigogine}
\author{}
\date{}
\maketitle

\section{Summary of Schr{\" o}dinger{'}s contributions}

By writing his book {``}What is life?{''} (1944) Schr{\" o}dinger inspired generations of physicists to answer the alluring (though not easy) question
of the role of physics in biological processes. In any event it probably wouldn{'}t be an exaggeration to say that Schr{\" o}dinger himself (as he
admits), was inspired by the work of German-American physicists Max Delbr{\" u}ck; who helped launch the molecular biology research program in the
late 1930s and explained (in main part) the mechanism of heredity and mutation.

Regardless, Schr{\" o}dinger makes some very essential observations on the nature of living organisms which I shall describe here shortly.

First, their operation (living organisms) as a macroscopic system resembles approximately, a purely mechanical system rather than a thermodynamical
system. Even though their size is far from what is considered a thermodynamic limit, they tend stay unaffected (in special environments) by random
molecular motion known as heat and; at the same time, evade the decay towards equilibrium for an unusually long time. This is essentially the definition
of a living system.

Secondly, he notices that the way an organism accomplishes the above is through the exchange of energy and matter with it's environment, that leaves
it's own internal state in low entropy. He withdraws from considerations of free energy, although he acknowledges that the exact physical understanding
should be accomplished through it rather than through entropy. Worth mentioning is his hypothesis of {``}life intensity{''} the term which ought
to parallel with the rate at which the system produces entropy.

Thirdly, each cell depends on very small group of atoms, the genetic code, which determine it's evolution, something unprecedented, beyond the description
of ordinary statistical physics. Perhaps, a partial explanation for this dynamical behavior (rather than statistical) can be traced to rigidity and
tightness of chemical bonds. However the very vital point Schr{\" o}dinger tries to make is the hypothesis, that there must exist a yet unknown,
new law of physics that would explain fully how order can be produced out of order. The formulation of this law is to my knowledge; still in development.

Lastly, even though Schr{\" o}dinger introduces some quantum mechanics principles, like the uniqueness of Heitler-London bond in order to defend
the theory laid down by Delbr{\" u}ck, he assures that quantum indeterminacy should play only marginal role in the future laws of dynamics of living
systems.

\section*{Summary of Prigogine contributions}

The term {``}dissipative structures{''} was first used by Ilya Prigogine (http://bactra.org/notebooks/dissipative-structures.html) and although Prigogine
ideas were not really correct ones it might be worth to recall some of them.

\subsection*{Nobel Lecture (8 December 1977)}

Prigogine nobel prize lecture {``}Time, structure and fluctuations{''} begins with the critique of Helmholtz free energy and the assertion that living
system posses a different type of functional order which can be traced to their non$--$equilibrium state. This statement is consistent with the today{'}s
predominant view.

{``}However thermodynamic potentials exist only for exceptional situations{''} - is that true?

One of his often cited contributions is connected with a term for entropy for open systems, an extension of Clausius entropy for isolated systems:

\(dS=d_iS+d_eS\)

Where \(d_iS\) is connected with entropy produced within the system and \(d_eS\) is the entropy transferred across the boundaries of the system.
The second law states that \(d_iS\geq 0\), so if a system is to stay in law entropy state it{'}s production must be compensated by an inflow of negative
entropy.

He then develops an explicit expression for entropy production, assuming that even outside equilibrium (but near) entropy depends only on the same
variables as at equilibrium ({``}local{''} equilibrium)

\(P=\frac{d_iS}{dt}=\sum _{\rho } J_{\rho }X_{\rho }\geq 0\)

where \(J_{\rho }\) are the rates of the various irreversible processes involved (chemical reactions, heat flow, diffusion$\ldots $) and \(X_{\rho
}\) are the corresponding, generalized forces (affinities, gradients of temperature, of chemical potentials$\ldots $). The flows are described using
different empirical laws (Fourier{'}s law, Fick{'}s law, etc.) 

\(J_{\rho }=\sum _{\rho } L_{\rho \rho '}X_{\rho '}\)

Onsager relations \(L_{\rho \rho '}=L_{\rho '\rho }\)

Prigogine rightly criticized the efforts to extend the principle of \pmb{ minimum entropy production (which is valid only very near equilibrium)
to non-equilibrium regimes, showing that }. What exactly is this principle is well explained by Prigogine; when a system is constricted by a boundary
condition and perturbed, the entropy production will increase, but then the system settles down to the state of {``}least dissipation{''}.\\
An example of a process such process, namely Rayleigh$--$B{\' e}nard convection is given, which he perceives as a prime example of occurrence of
{``}dissipative structures{''} which fail to be described by Boltzmann laws. In Prigogine view the fluctuations are the trigger for the instabilities
instabilities, which in turn give rise to spacetime structure. Here instabilities carry the sense of bifurcations of equations of motion.

I need to learn more about Nicolis work on dynamics of chemical reactions

Furthermore Prigogine develops an uncommon perspective on the microscopic equations of motion, which in his opinion should not be invariant under
time inversion.\\
A proposed way to achieve this is through a non$--$unitary transformation which yield a type Lyapounov function, analogue to Bolzmann H-function.
The goal was to obtain a microscopic representation of entropy. It{'}s known in literature as Misra-Prigogine-Courbage theory of irreversibility.

Prigogine view on reversibility are probably best summarized by the quotes {``}I have always found it difficult to accept this conclusion [macroscopic
irreversibility emerging from initial conditions] { }especially because of the constructive role of irreversible processes. Can dissipative structures
be the result of mistakes?{''}

At the present moment I can{'}t comment much on this, but some extensive critique can be found {``}Science of Chaos or Chaos in Science{''} by Bricmont

The best account (or the most understandable) of Prigogine views are perhaps his own words in {``}Laws of Chaos{''} - {``}The essential condition
is that the microscopic description of the universe be made in terms of unstable dynamical systems. This is a radical change in point of view. From
the point of view of classical physics, stable systems were the rule and unstable systems the exceptions. We are now reversing that perspective.{''}\\
First it{'}s not true that stable systems are the rule in classical physics, one can easily devise classical examples of unstable systems following
purely classical mechanics, the reason why they are not being analyzed is that analytic methods fail to solve them. Prigogine seems to require more
complexity to explain complexity, which is in my opinion not needed.

\end{document}
